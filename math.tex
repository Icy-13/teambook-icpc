\begin{enumerate}
    \item \textbf{Функция Мёбиуса.}  
    \[
        \mu(n) =
        \begin{cases}
            (-1)^k, & n = p_1 p_2 \cdots p_k \ \text{--- произведение $k$ различных простых}, \\
            0, & n \ \text{делится на квадрат простого}.
        \end{cases}
    \]  
    \textbf{Обращение Мёбиуса:} если 
    \[
        g(n) = \sum_{d \mid n} f(d),
    \]
    то
    \[
        f(n) = \sum_{d \mid n} \mu(d)\, g\!\left(\tfrac{n}{d}\right).
    \]

    $\sum_{d|n} \mu(d) = \mathbf{I}(n = 1)$ индикатор что $n=1$. Можно проверять предикаты

    Посчитаем пары взаимно-простых в $[1, n]$: $\sum_{i,j=1}^{n}[gcd(i, j)=1] = \sum_{i,j=1}^{n}\sum_{d|gcd(i, j)}\mu(d)=$ 
    
    $\sum_{i,j=1}^{n}\sum_{d=1}^{n}[d|gcd(i, j)]\mu(d) =$
    $ \sum_{i,j=1}^{n}\sum_{d=1}^{n} [d|i][d|j]\mu(d) =$ 
    
    $\sum_{d=1}^{n}\mu(d) (\lfloor \frac{n}{d} \rfloor)^2$

    \item \textbf{Лемма Бёрнсайда.} Чтобы посчитать классы эквивалентности, нужно для каждой валидной перестановки посчитать количество неподвижных точек, сложить это, и поделить на количество перестановок:
    \[
        Classes = \frac{1}{|G|} \sum_{g \in G} | \mathrm{Fix}(g) |.
    \]  

    \textbf{Теорема Пойя.} Если каждый элемент независимо от других принимает одно из $k$ состояний, то:
    \[
        Classes= \frac{1}{|G|} \sum_{g \in G} k^{c(g)},
    \]
    где $c(g)$ --- число циклов в перестановке $g$.

    \item \textbf{Лемма Холла.} Для двудольного графа $G=(X,Y,E)$ существует совершенное паросочетание $\Leftrightarrow$ для любого $S \subseteq X$ выполняется
    \[
        |N(S)| \geq |S|.
    \]

    \item \textbf{Длина кривой.} Пусть $\gamma(t) = (x(t), y(t))$, $t \in [a,b]$, гладкая кривая. Тогда её длина:
    \[
        L = \int_a^b \sqrt{ (x'(t))^2 + (y'(t))^2 } \, dt.
    \]

    \item \textbf{Формула Симпсона.} Для численного интегрирования:
    \[
        \int_a^b f(x)\, dx \approx \frac{b-a}{6}\Big( f(a) + 4 f\!\left(\tfrac{a+b}{2}\right) + f(b) \Big).
    \]

    \item \textbf{Числа Каталана.} 1, 1, 2, 5, 14, 42, 132, 429, 1430, 4862... $a(n)$ is odd if and only if $n=2^{k}-1$
    \[
        C_n = \frac{1}{n+1}\binom{2n}{n}, \, C_{n+1} = \sum_{k=0}^n C_k C_{n-k}.
    \]
    
    \item \textbf{Числа Стирлинга первого рода.} Количество перестановок порядка $n$ с $k$ циклами
    \[
        \left[ {n \atop k} \right] = \left[ {n-1 \atop k-1} \right] + (n-1) \cdot \left[ {n-1 \atop k} \right], \, 
        \left[ {0 \atop 0} \right] = 1
    \]

    \item \textbf{Числа Стирлинга второго рода.} Количество способов разбить $n$-элементное множество на $k$ непустых подмножеств
    \[
        \left\{ {n \atop k} \right\} = k \left\{ {n-1 \atop k} \right\} + \left\{ {n-1 \atop k-1} \right\}, \, 
        \left\{ {0 \atop 0} \right\} = 1.
    \]

    \item \textbf{Числа Эйлера (первого рода).} Количество перестановок порядка $n$ с $k$ подъемами. $A(0, 1)=1$
    \[
        A(n,k) = (n-k)A(n-1, k-1) + (k+1)A(n-1,k)
    \]

    \item \textbf{Числа Белла} Количество способов разбиения множества из $n$ элементов на подмножества. Сумма чисел Стирлинга второго рода. 1, 1, 2, 5, 15, 52, 203, 877, 4140... 

    \[
        B_{n+1} = \sum_{k=0}^{n} \binom{n}{k} B_{k}
    \]

    \item \textbf{Цепи Маркова} TODO
    \item \textbf{Функция Эйлера} $a^{\phi(m)} \equiv 1 (mod \, m)$ \\
    $ a^{n} \equiv a^{n \, mod \, \phi(m)} (mod \, m) \quad \sum_{d|n} \phi(d) = n$

    \item \textbf{Количество различных подстрок}  
    \[
        ans = \frac{n(n+1)}{2} - \sum_{i=0}^{n-2} lcp[i]
    \]

    \item \textbf{Переходы в Фенвике} r += r \& -r для обновления, r -= r \& -r для суммы, 1-индексация

    \item \textbf{Матожидание размера выпуклой оболочки на случайных точках:} $O(\log n)$ для плоскости, $O(n^{\frac{1}{3}})$ для окружности

\end{enumerate}
