\begin{enumerate}
    \item \textbf{Функция Мёбиуса.}  
    \[
        \mu(n) =
        \begin{cases}
            (-1)^k, & n = p_1 p_2 \cdots p_k \ \text{--- произведение $k$ различных простых}, \\
            0, & n \ \text{делится на квадрат простого}.
        \end{cases}
    \]  
    \textbf{Обращение Мёбиуса:} если 
    \[
        g(n) = \sum_{d \mid n} f(d),
    \]
    то
    \[
        f(n) = \sum_{d \mid n} \mu(d)\, g\!\left(\tfrac{n}{d}\right).
    \]

    $\sum_{d|n} \mu(d) = \mathbf{I}(n = 1)$ индикатор что $n=1$. Можно проверять предикаты

    Посчитаем пары взаимно-простых в $[1, n]$: $\sum_{i,j=1}^{n}[gcd(i, j)=1] = \sum_{i,j=1}^{n}\sum_{d|gcd(i, j)}\mu(d)=$ 
    
    $\sum_{i,j=1}^{n}\sum_{d=1}^{n}[d|gcd(i, j)]\mu(d) =$
    $ \sum_{i,j=1}^{n}\sum_{d=1}^{n} [d|i][d|j]\mu(d) =$ 
    
    $\sum_{d=1}^{n}\mu(d) (\lfloor \frac{n}{d} \rfloor)^2$

    \item \textbf{Лемма Бёрнсайда.} Чтобы посчитать классы эквивалентности, нужно для каждой валидной перестановки посчитать количество неподвижных точек, сложить это, и поделить на количество перестановок:
    \[
        Classes = \frac{1}{|G|} \sum_{g \in G} | \mathrm{Fix}(g) |.
    \]  

    \textbf{Теорема Пойя.} Если каждый элемент независимо от других принимает одно из $k$ состояний, то:
    \[
        Classes= \frac{1}{|G|} \sum_{g \in G} k^{c(g)},
    \]
    где $c(g)$ --- число циклов в перестановке $g$.

    \item \textbf{Лемма Холла.} Для двудольного графа $G=(X,Y,E)$ существует совершенное паросочетание $\Leftrightarrow$ для любого $S \subseteq X$ выполняется
    \[
        |N(S)| \geq |S|.
    \]

    \item \textbf{Длина кривой.} Пусть $\gamma(t) = (x(t), y(t))$, $t \in [a,b]$, гладкая кривая. Тогда её длина:
    \[
        L = \int_a^b \sqrt{ (x'(t))^2 + (y'(t))^2 } \, dt.
    \]

    \item \textbf{Формула Симпсона.} Для численного интегрирования:
    \[
        \int_a^b f(x)\, dx \approx \frac{b-a}{6}\Big( f(a) + 4 f\!\left(\tfrac{a+b}{2}\right) + f(b) \Big).
    \]

    \item \textbf{Числа Каталана.} 1, 1, 2, 5, 14, 42, 132, 429, 1430, 4862... $a(n)$ is odd if and only if $n=2^{k}-1$
    \[
        C_n = \frac{1}{n+1}\binom{2n}{n}, \, C_{n+1} = \sum_{k=0}^n C_k C_{n-k}.
    \]
    
    \item \textbf{Числа Стирлинга первого рода.} Количество перестановок порядка $n$ с $k$ циклами
    \[
        \left[ {n \atop k} \right] = \left[ {n-1 \atop k-1} \right] + (n-1) \cdot \left[ {n-1 \atop k} \right], \, 
        \left[ {0 \atop 0} \right] = 1
    \]

    \item \textbf{Числа Стирлинга второго рода.} Количество способов разбить $n$-элементное множество на $k$ непустых подмножеств
    \[
        \left\{ {n \atop k} \right\} = k \left\{ {n-1 \atop k} \right\} + \left\{ {n-1 \atop k-1} \right\}, \, 
        \left\{ {0 \atop 0} \right\} = 1.
    \]

    \item \textbf{Числа Эйлера (первого рода).} Количество перестановок порядка $n$ с $k$ подъемами. $A(0, 1)=1$
    \[
        A(n,k) = (n-k)A(n-1, k-1) + (k+1)A(n-1,k)
    \]

    \item \textbf{Числа Белла} Количество способов разбиения множества из $n$ элементов на подмножества. Сумма чисел Стирлинга второго рода. 1, 1, 2, 5, 15, 52, 203, 877, 4140... 

    \[
        B_{n+1} = \sum_{k=0}^{n} \binom{n}{k} B_{k}
    \]

    \item \textbf{Цепи Маркова} TODO
    \item \textbf{Функция Эйлера} $a^{\phi(m)} \equiv 1 (mod \, m)$ \\
    $ a^{n} \equiv a^{n \, mod \, \phi(m)} (mod \, m) \quad \sum_{d|n} \phi(d) = n$

    \item \textbf{Количество различных подстрок}  
    \[
        ans = \frac{n(n+1)}{2} - \sum_{i=0}^{n-2} lcp[i]
    \]

    \item \textbf{Переходы в Фенвике} r += r \& -r для обновления, r -= r \& -r для суммы, 1-индексация
    \item \textbf{2-SAT.} Для каждой переменной добавить вершину с ней и отрицанием, построить граф импликаций, выделить в нем КСС. Если x и !x в одной компоненте, то нельзя. Иначе значения можно получить как comp[i] > comp[i + 1]

    \item \textbf{Матожидание размера выпуклой оболочки на случайных точках:} $O(\log n)$ для плоскости, $O(n^{\frac{1}{3}})$ для окружности
    \item \textbf{Segment Tree Beats} Общая идея - усилим break condition и ослабим tag. Базовые ифы сохраняются.
    \begin{enumerate}
        \item \%=, = на отрезке, get ;; break: maxvalue[node] < mod ;; tag: maxvalue[node]==minvalue[node]
        \item min=, max=, += $\sum$, min, max, gcd ;; break: max <= x ;; tag: secondMax < x
        \item $\sqrt{}$=, /=, +=, sum, min, max ;; break: default ;; tag: $max[v] - min[v] \le 1$ 
        \item \&=, |=, max ;; break: default ;; tagAnd: ((and[v] $\oplus$ or[v]) \& $\sim$ x)==0 ;; tagOr: ((and[v] $\oplus$ or[v]) \& x)==0
    \end{enumerate}

    \item \textbf{Гипотеза Гольдбаха} любое \textbf{четное} число представимо в виде суммы двух простых
    \item \textbf{Теорема Шпрага-Гранди}. Число Гранди - mex чисел Гранди состояний в которые мы можем прийти, или ноль если никуда не можем. Если ноль, то состояние проигрышное. Если переход ведет в сумму игр, то числа Гранди из каждой игры надо будет поксорить
    
    Пример: есть массив длинной $n$, каждый ход ставят крестик в клетку, в которой нет креста и у ее соседей нет креста. Тогда каждый ход разбивает игру на две - слева и справа от креста
    
    $g(n) = mex(\{g(n-2)\} \cup \{ g(i - 2) \oplus g(n-i-1) | 2 \le i \le n-1 \})$

    \item \[ \sum_{k=0}^{n} \binom{n-k}{k} = Fib_{n+1} ~~~ \sum_{k=0}^{r} \binom{m}{k} \binom{n}{r-k} = \binom{m+n}{r}\]
    \item \[ \sum_{i=r}^{n} \binom{i}{r} = \binom{n+1}{r+1} ~~~ \sum_{i=0}^n \binom{n}{i}^2 = \binom{2n}{n} \]
    \item \textbf{Derangement} перестановка где все элементы не на оригинальной позиции $d(n) = (n-1) \cdot (d(n-1) + d(n-2)) ~~~ d(0) = 1, d(1) = 0$
    \item \textbf{Теорема Лукаса} если $p$ простое, то $\frac{p^{\alpha}}{k} \equiv 0 (mod ~ p)$
    
    Если $p$ простое и $m, n \ge 0$, то $\frac{m}{n} \equiv \prod^{k}_{i=0} \frac{m_i}{n_i} (mod ~ p)$, где $m = m_k p^k + m_{k-1}p^{k-1} + ...$

    \item \[ gcd(F_m, F_n) = F_{gcd(m, n)} ~~ \sum_{d|n} \frac{\mu(d)^2}{\phi(d)} = \frac{n}{\phi(n)}\]
    \item mex(a[l:r]) = min(mex(a[0:r]), mex(a[l:n]))
    \item \textbf{Алгоритм Мо:} посортим блоки запросов по левой границе, внутри блока по правой.
    \item \textbf{3D Мо:} $t$=количество апдейтов до текущего запроса. Отсортируем в порядке $(\lfloor \frac{t}{n^{\frac{2}{3}}} \rfloor, \lfloor \frac{l}{n^{\frac{2}{3}}} \rfloor, r)$. $O(n^{\frac{5}{3}})$

    \item \textbf{Известные неравенства.}
    \begin{enumerate}
        \item \textbf{КБШ:}
        \[
            \left(\sum_{i=1}^{n} a_i b_i\right)^2 \leq \left(\sum_{i=1}^{n} a_i^2\right)\left(\sum_{i=1}^{n} b_i^2\right)
        \]
        Равенство $\Leftrightarrow$ векторы коллинеарны.
        
        Форма через скалярное произведение: $|\langle \vec{a}, \vec{b} \rangle|^2 \leq \langle \vec{a}, \vec{a} \rangle \cdot \langle \vec{b}, \vec{b} \rangle$ или $|\langle \vec{a}, \vec{b} \rangle| \leq \|\vec{a}\| \cdot \|\vec{b}\|$
        
        \item \textbf{О средних:} для $a_i > 0$:
        \[
            \text{QM} \geq \text{AM} \geq \text{GM} \geq \text{HM}
        \]
        где
        \begin{align*}
            \text{QM} &= \sqrt{\frac{a_1^2 + a_2^2 + \cdots + a_n^2}{n}} \\
            \text{AM} &= \frac{a_1 + a_2 + \cdots + a_n}{n} \\
            \text{GM} &= \sqrt[n]{a_1 a_2 \cdots a_n} \\
            \text{HM} &= \frac{n}{\frac{1}{a_1} + \frac{1}{a_2} + \cdots + \frac{1}{a_n}}
        \end{align*}
        Равенство в любом из них $\Leftrightarrow$ все $a_i$ равны.
        
        \item \textbf{Гёльдер:} для $p, q > 1$ с $\frac{1}{p} + \frac{1}{q} = 1$:
        \[
            \sum_{i=1}^{n} |a_i b_i| \leq \left(\sum_{i=1}^{n} |a_i|^p\right)^{1/p} \left(\sum_{i=1}^{n} |b_i|^q\right)^{1/q}
        \]
        
        \item \textbf{Минковский:} для $p \geq 1$:
        \[
            \left(\sum_{i=1}^{n} |a_i + b_i|^p\right)^{1/p} \leq \left(\sum_{i=1}^{n} |a_i|^p\right)^{1/p} + \left(\sum_{i=1}^{n} |b_i|^p\right)^{1/p}
        \]
        
        \item \textbf{Йенсен:} если $f$ выпуклая функция, то:
        \[
            f\left(\frac{x_1 + x_2 + \cdots + x_n}{n}\right) \leq \frac{f(x_1) + f(x_2) + \cdots + f(x_n)}{n}
        \]
        
        
        \item \textbf{Бернулли:} для $x > -1$ и $n \in \mathbb{N}$:
        \[
            (1 + x)^n \geq 1 + nx
        \]
        
        \item \textbf{Марков:} для неотрицательной с.в. $X$ и $a > 0$:
        \[
            P(X \geq a) \leq \frac{E[X]}{a}
        \]
        
        \item \textbf{Чебышёв:} с.в. $X$ с $E[X] = \mu$, $\text{Var}(X) = \sigma^2$:
        \[
            P(|X - \mu| \geq k\sigma) \leq \frac{1}{k^2} \quad \text{или} \quad P(|X - \mu| \geq \varepsilon) \leq \frac{\sigma^2}{\varepsilon^2}
        \]
        
        \item \textbf{Чернова:} для $X = \sum_{i=1}^n X_i$, где $X_i$ независимы, $X_i \in \{0,1\}$, $\mu = E[X]$:
        \[
            P(X \geq (1+\delta)\mu) \leq e^{-\frac{\delta^2 \mu}{3}}, \quad \delta > 0
        \]
        \[
            P(X \leq (1-\delta)\mu) \leq e^{-\frac{\delta^2 \mu}{2}}, \quad 0 < \delta < 1
        \]
        
        \item \textbf{Йенсен} если $f$ выпуклая:
        \[
            f(E[X]) \leq E[f(X)]
        \]
        
        \item \textbf{КБШ (для с.в.):}
        \[
            |E[XY]|^2 \leq E[X^2] \cdot E[Y^2]
        \]
        Следствие: $|\text{Cov}(X,Y)|^2 \leq \text{Var}(X) \cdot \text{Var}(Y)$
    \end{enumerate}

    \item \textbf{Таблица производных.}
    \begin{align*}
        (a^x)' &= a^x \ln a & (e^x)' &= e^x \\
        (\log_a x)' &= \frac{1}{x \ln a} & (\sin x)' &= \cos x \\
        (\cos x)' &= -\sin x & (\tan x)' &= \frac{1}{\cos^2 x} \\
        (\cot x)' &= -\frac{1}{\sin^2 x} & (\arcsin x)' &= \frac{1}{\sqrt{1-x^2}} \\
        (\arccos x)' &= -\frac{1}{\sqrt{1-x^2}} & (\arctan x)' &= \frac{1}{1+x^2} \\
        (\text{arccot}\, x)' &= -\frac{1}{1+x^2}
    \end{align*}

    \item \textbf{Таблица интегралов.}
    {\small
    \begin{align*}
        \int x^n dx &= \frac{x^{n+1}}{n+1},\, n \neq -1 & 
        \int \frac{dx}{x} &= \ln|x| \\
        \int e^x dx &= e^x & 
        \int a^x dx &= \frac{a^x}{\ln a} \\
        \int \sin x dx &= -\cos x & 
        \int \cos x dx &= \sin x \\
        \int \frac{dx}{\cos^2 x} &= \tan x & 
        \int \frac{dx}{\sin^2 x} &= -\cot x \\
        \int \frac{dx}{1+x^2} &= \arctan x & 
        \int \frac{dx}{\sqrt{1-x^2}} &= \arcsin x \\
        \int \frac{dx}{x^2+a^2} &= \frac{1}{a}\arctan\frac{x}{a} & 
        \int \frac{dx}{\sqrt{a^2-x^2}} &= \arcsin\frac{x}{a} \\
        \int \frac{dx}{\sqrt{x^2 \pm a^2}} &= \ln|x + \sqrt{x^2 \pm a^2}| &
        \int \tan x dx &= -\ln|\cos x|
    \end{align*}
    }

    \item \textbf{Сложные интегралы.}
    {\small
    \begin{align*}
        \int \ln x dx &= x\ln x - x \\
        \int \frac{xdx}{\sqrt{x^2 \pm a^2}} &= \sqrt{x^2 \pm a^2} \\
        \int \frac{dx}{x^2-a^2} &= \frac{1}{2a}\ln\left|\frac{x-a}{x+a}\right| \\
        \int \frac{dx}{a^2-x^2} &= \frac{1}{2a}\ln\left|\frac{a+x}{a-x}\right| \\
        \int \sqrt{a^2-x^2}\, dx &= \frac{x}{2}\sqrt{a^2-x^2} + \frac{a^2}{2}\arcsin\frac{x}{a} \\
        \int \sqrt{x^2 \pm a^2}\, dx &= \frac{x}{2}\sqrt{x^2 \pm a^2} \pm \frac{a^2}{2}\ln|x + \sqrt{x^2 \pm a^2}| \\
    \end{align*}
    \vspace{-5mm}
    }

    \item \textbf{Тригонометрические тождества.}
    {\small
    \begin{align*}
        1 + \tan^2 x &= \frac{1}{\cos^2 x} &
        1 + \cot^2 x &= \frac{1}{\sin^2 x} \\
        \sin 2x &= 2\sin x\cos x &
        \cos 2x &= \cos^2 x - \sin^2 x \\
        \cos 2x &= 2\cos^2 x - 1 &
        \cos 2x &= 1 - 2\sin^2 x \\
        \tan 2x &= \frac{2\tan x}{1-\tan^2 x} &
        \sin^2 x &= \frac{1-\cos 2x}{2} \\
        \cos^2 x &= \frac{1+\cos 2x}{2} \\
    \end{align*}
    \vspace{-15mm}
    \begin{align*}
        \sin(x \pm y) &= \sin x\cos y \pm \cos x\sin y \\
        \cos(x \pm y) &= \cos x\cos y \mp \sin x\sin y \\
        \tan(x \pm y) &= \frac{\tan x \pm \tan y}{1 \mp \tan x\tan y} \\
        \sin x + \sin y &= 2\sin\frac{x+y}{2}\cos\frac{x-y}{2} \\
        \sin x - \sin y &= 2\cos\frac{x+y}{2}\sin\frac{x-y}{2} \\
        \cos x + \cos y &= 2\cos\frac{x+y}{2}\cos\frac{x-y}{2} \\
        \cos x - \cos y &= -2\sin\frac{x+y}{2}\sin\frac{x-y}{2} \\
        \sin x\sin y &= \frac{1}{2}[\cos(x-y)-\cos(x+y)] \\
        \cos x\cos y &= \frac{1}{2}[\cos(x-y)+\cos(x+y)] \\
        \sin x\cos y &= \frac{1}{2}[\sin(x+y)+\sin(x-y)]
    \end{align*}
    }

\end{enumerate}
